%%%%%%%%%%%%%%%%%%%%%%%%%%%%%%%%%%%%%%%%%
% Beamer Presentation
% LaTeX Template
% Version 1.0 (10/11/12)
%
% This template has been downloaded from:
% http://www.LaTeXTemplates.com
%
% License:
% CC BY-NC-SA 3.0 (http://creativecommons.org/licenses/by-nc-sa/3.0/)
%
%%%%%%%%%%%%%%%%%%%%%%%%%%%%%%%%%%%%%%%%%

%----------------------------------------------------------------------------------------
%	PACKAGES AND THEMES
%----------------------------------------------------------------------------------------

\documentclass{beamer}

\mode<presentation> {

% The Beamer class comes with a number of default slide themes
% which change the colors and layouts of slides. Below this is a list
% of all the themes, uncomment each in turn to see what they look like.

%\usetheme{default}
%\usetheme{AnnArbor}
%\usetheme{Antibes}
%\usetheme{Bergen}
%\usetheme{Berkeley}
%\usetheme{Berlin}
%\usetheme{Boadilla}
%\usetheme{CambridgeUS}
%\usetheme{Copenhagen}
%\usetheme{Darmstadt}
%\usetheme{Dresden}
%\usetheme{Frankfurt}
%\usetheme{Goettingen}
%\usetheme{Hannover}
%\usetheme{Ilmenau}
%\usetheme{JuanLesPins}
%\usetheme{Luebeck}
\usetheme{Madrid}
%\usetheme{Malmoe}
%\usetheme{Marburg}
%\usetheme{Montpellier}
%\usetheme{PaloAlto}
%\usetheme{Pittsburgh}
%\usetheme{Rochester}
%\usetheme{Singapore}
%\usetheme{Szeged}
%\usetheme{Warsaw}

% As well as themes, the Beamer class has a number of color themes
% for any slide theme. Uncomment each of these in turn to see how it
% changes the colors of your current slide theme.

%\usecolortheme{albatross}
%\usecolortheme{beaver}
%\usecolortheme{beetle}
%\usecolortheme{crane}
%\usecolortheme{dolphin}
%\usecolortheme{dove}
%\usecolortheme{fly}
%\usecolortheme{lily}
%\usecolortheme{orchid}
%\usecolortheme{rose}
%\usecolortheme{seagull}
%\usecolortheme{seahorse}
%\usecolortheme{whale}
%\usecolortheme{wolverine}

%\setbeamertemplate{footline} % To remove the footer line in all slides uncomment this line
%\setbeamertemplate{footline}[page number] % To replace the footer line in all slides with a simple slide count uncomment this line

%\setbeamertemplate{navigation symbols}{} % To remove the navigation symbols from the bottom of all slides uncomment this line
}

\usepackage{graphicx} % Allows including images
\usepackage{booktabs} % Allows the use of \toprule, \midrule and \bottomrule in tables

%----------------------------------------------------------------------------------------
%	TITLE PAGE
%----------------------------------------------------------------------------------------

\title[Short title]{DISCUSSION LOG} % The short title appears at the bottom of every slide, the full title is only on the title page

\author{Sikang Yan} % Your name
\institute[TUK] % Your institution as it will appear on the bottom of every slide, may be shorthand to save space
{
University of Kaiserslautern \\ % Your institution for the title page
\medskip
\textit{yan@rhrk.uni-kl.de} % Your email address
}
\date{\today} % Date, can be changed to a custom date

\begin{document}

\begin{frame}
\titlepage % Print the title page as the first slide
\end{frame}

%------------------------------------------------

\begin{frame}
\frametitle{KW13}
In our cloth simulation, we follow the precedure written by Rohmer et al..


We consider to begin with the \emph{Defomation gradient} $\mathbf{F}$, which is defined by
\begin{align}
\mathbf{F} = \dfrac{\partial\mathbf{x}}{\partial\mathbf{X}},  
\end{align}
where the $\mathbf{x}$ denotes the deformed vector and the $\mathbf{X}$ denotes the reference vector.


Since we have no further information about the mapping from $\mathbf{X}$ to 
$\mathbf{x}$, we use the vector $\big(\mathbf{u_1},\mathbf{u_2}\big)$ and $\big(\overline{\mathbf{u_1}},\overline{\mathbf{u_2}}\big)$ to approximate the Defomation gradient, which is defined as
\begin{align}
\mathbf{F} = 
\big[\mathbf{u_1},\mathbf{u_2}\big]
\big[\overline{\mathbf{u_1}},\overline{\mathbf{u_2}}\big]^{-1}, 
\label{eq:Defomation_gradient_el} 
\end{align}

Attention should be paid especially:
\begin{itemize}
\item eq.\eqref{eq:Defomation_gradient_el} characrize only the 2D deformation of each triangle.
\item in Rohmer et al. $\mathbf{F}$ is symbolised as $\mathbf{T}$.
\end{itemize}
\end{frame}

%------------------------------------------------

\begin{frame}
\frametitle{KW13}
We provide here our code preceed with concrete data:
\begin{itemize}
\item $faces(i,j)$ is the $j$th vertex of the $i$th triangle, here we choose the $face(0,0),face(0,1),face(0,2)$ as examples. 
\end{itemize}
\end{frame}

%------------------------------------------------

\begin{frame}
\begin{block}{$cloth\_vec$}
\begin{align}
VecT=&el_1VertT_1-el_1VertT_2;el_1VertT_1-el_1VertT_3 \\ 
VecR=&el_1VertR_1-el_1VertR_2;el_1VertR_1-el_1VertR_3 
\end{align}
\end{block}
\begin{block}{$cloth\_vec$}
\begin{align}
VecT=&[ 0.771842 \ -0.0144887 \ 6.39045 ]-[ 0.780121 \ -0.0186188 \ 6.37318 ];\\
     &[ 0.771842 \ -0.0144887 \ 6.39045 ]-[ 0.737177\ -0.00912791 \ 6.39292 ] \\
VecR=&[ 0.759919 \ -0.015194 \ 6.38401 ]-[ 0.767822 \ -0.0212492 \ 6.3669  ];\\
	 &[ 0.759919 \-0.015194  \ 6.38401 ]-[ 0.726977 \ -0.00749985 \ 6.38692 ] \\
\end{align}
\end{block}
\end{frame}

%----------------------------------------------------------------------------------------

\begin{frame}
such that
\begin{block}{$cloth\_vec$}
\begin{align}
VecT=&[ -0.0079  \  0.0061 \  0.0171 \ 0.0329 \ -0.0077 \ -0.0029  ];\\ 
VecR=&[ -0.0083  \  0.0041 \  0.0173 \ 0.0347 \ -0.0054 \ -0.0025  ];\\
\end{align}
\end{block} 
where the first 3 entries of $VecR$ is the vector $\mathbf{u_1}$ and the last 3 entries of $VecT$ is the vector $\mathbf{u_2}$. $VecT$ analogiously.
\end{frame}

%----------------------------------------------------------------------------------------

\begin{frame}
\begin{block}{$cloth\_eig\_2D$}
we use here $Eigen::Map$ to transform the vector $VecT$ and $VecR$ to $2*2$ 2D deformation gradient $\mathbf{F}$.
\begin{align}
\mathbf{F}
=
\biggl[
\begin{matrix}
   -0.0083 & 0.0347 \\
   0.0041 & -0.0054 \\
\end{matrix}
\biggr]
\biggl[
\begin{matrix}
   -0.0079 & 0.0329 \\
   0.0061 & -0.0077 \\
\end{matrix}
\biggr]^{-1}
\end{align} 
\end{block} 
\end{frame}

%----------------------------------------------------------------------------------------

\begin{frame}
hence the $\mathbf{F}$ has a rotation information $\mathbf{R}$ and a stretch information $\mathbf{U}$,
\begin{align}
\mathbf{F}=\mathbf{R}\mathbf{U}.
\end{align}
we use $\mathbf{F}^T\mathbf{F}$ to eliminate the ratotion information to obtain $\det\mathbf{R}=1$ 
\begin{align}
\mathbf{F}^T\mathbf{F}=&
\bigl( \mathbf{R}\mathbf{U} \bigr) ^T\mathbf{R}\mathbf{U} \\
=&\mathbf{U}^T\mathbf{R}^T\mathbf{R}\mathbf{U} \\
=&\mathbf{U}^2 \\
=&\mathbf{C}
\end{align}
and using decomposition, if we have the form
\begin{align}
\mathbf{C} = \lambda_1^2\mathbf{v}_1\mathbf{v}_1^T+\lambda_2^2\mathbf{v}_2\mathbf{v}_2^T
\end{align} 
then we could obtain the $\mathbf{U}$ by applying
\begin{align}
\mathbf{U} = \lambda_1\mathbf{v}_1\mathbf{v}_1^T+\lambda_2\mathbf{v}_2\mathbf{v}_2^T
\end{align}
\end{frame}


\end{document} 